%%% Globale Einstellungen und Laden von Paketen (~Bibliotheken)
\documentclass[aspectratio=169,presentation]{beamer}
%\documentclass[aspectratio=169,handout]{beamer}

\usetheme{Boadilla} % Bestimmt das gesamte Erscheinungsbild, die folgenden fand ich grundsätzlich ganz passend:
% Hannover, Singapore, Malmoe, Boadilla, CambridgeUS

\useinnertheme{default}

% setting locales
\usepackage[utf8]{inputenc}
\usepackage[T1]{fontenc}
\usepackage[ngerman]{babel}
\usepackage{lmodern}
\usepackage[locale=DE,mode=math,list-final-separator={ oder },range-phrase={ bis },scientific-notation=false,group-digits=integer]{siunitx}

% package includes
\usepackage{tikz}
\usetikzlibrary{positioning,automata}
\usetikzlibrary{arrows,shapes}
\usetikzlibrary{fit}
\usepackage{verbatim}
\usepackage{xcolor}
\usepackage{listings}

%listing setup
\definecolor{pblue}{rgb}{0.13,0.13,1}
\definecolor{pgreen}{rgb}{0,0.5,0}
\definecolor{pred}{rgb}{0.9,0,0}
\definecolor{pgrey}{rgb}{0.46,0.45,0.48}
\definecolor{javared}{rgb}{0.6,0,0} % for strings
\definecolor{javagreen}{rgb}{0.25,0.5,0.35} % comments
\definecolor{javapurple}{rgb}{0.5,0,0.35} % keywords
\definecolor{javadocblue}{rgb}{0.25,0.35,0.75} % javadoc

\lstset{language=vhdl,
	basicstyle=\ttfamily,
	keywordstyle=\color{javapurple}\bfseries,
	stringstyle=\color{javared},
	commentstyle=\color{javagreen},
	morecomment=[s][\color{javadocblue}]{/**}{*/},
	tabsize=2,
	showspaces=false,
	showstringspaces=false
}


%%% (Wahrscheinlich ziemlich dreckige) Umsetzung von Spezialframes, die nur groß den Titel beinhalten
\newcommand{\sectionframe}[1]{
	\begin{frame}
		\vfill
		\Huge
		\centering
		\usebeamercolor[fg]{title}
		#1
		\vfill
		\par
	\end{frame}
}



%%% Zentrales Festelegen von Terminnummer und Datum
\newcommand{\terminNummer}{5}
\date{\today}
%%%



\begin{document}
\title[CE Tutorium]{Tutorium zu\\Computer-Engineering\\im SS19}
\subtitle{Termin \terminNummer}
\author[Otto]{Jakob Otto}
\institute{HAW Hamburg}
\subject{CE Tutorium}
\pgfdeclareimage[height=0.5cm]{university-logo}{logo-haw-2017}
\logo{\href{http://haw-hamburg.de}{\pgfuseimage{university-logo}}}

\titlepage

%---------------------------------------------------------------------------------------------------------------------
%	Ablauf
%---------------------------------------------------------------------------------------------------------------------
\section{Was steht an?}
\begin{frame}{Ablauf}
	\begin{columns}
		\column{0.6\textwidth}
    \begin{itemize}
      \item Statemachines in VHDL
      \item VHDL-teil Aufgabe 5
		\end{itemize}
		\column{0.4\textwidth}
		\includegraphics[width=0.6\textwidth]{kratzen}
	\end{columns}
\end{frame}

%---------------------------------------------------------------------------------------------------------------------
%	Ideen für Aufgabe 1
%---------------------------------------------------------------------------------------------------------------------

\section{4 Phasen handshake in VHDL}
\begin{frame} {4 Phasen handshake Controller}
  \begin{itemize}
    \item STM-32-Seite letztes mal geklärt
    \begin{itemize}
      \item Wie also nun FPGA-Seite?
    \end{itemize}
    \item 4 Phasen Handshake gut durch Statemachine darstellbar
    \item 4 Zustände
    \item Lesen/Schreiben nicht großartig unterschiedlich
    \begin{itemize}
      \item Jedenfalls nicht auf FPGA Seite
    \end{itemize}
  \end{itemize}
\end{frame}


\begin{frame} {4 Phasen Statemachine}
  \tikzstyle{format} = [draw, thin, fill=blue!20]
  \tikzstyle{medium} = [ellipse, draw, thin, fill=green!20, minimum height=2.5em]
  
  \begin{center}
    \begin{figure}
      \begin{tikzpicture}[node distance=3cm, auto,>=latex', thick]
        \node[state,initial]   (A)                {$P_1$};
        \node[state]           (B) [right=of A]   {$P_2$};
        \node[state]           (C) [right=of B]   {$P_3$};
        \node[state]           (D) [right=of C]   {$P_4$};
        \path[->] (A) edge node [above] {$REQ$} (B)
                  (B) edge node [above] {$ACK$} (C)
                  (C) edge node [above] {$\overline{REQ}$} (D)
                  (D) edge [bend right=30] node [above] {$\overline{ACK}$} (A);
      \end{tikzpicture}
    \end{figure}
  \end{center}
\end{frame}




\begin{frame} [fragile] {Statemachine Aufbau I}
  \begin{lstlisting}
-- 2 bit - 4 states
signal state_ns : std_logic_vector(1 downto 0);
signal state_cs : std_logic_vector(1 downto 0) := (others=>'0');

-- 4 phase handshake 
signal req_cs : std_logic := '0';

signal ack_ns : std_logic;
signal ack_cs : std_logic := '0';

signal rnw_cs : std_logic := '0';
  \end{lstlisting}
\end{frame}


\begin{frame} [fragile] {Statemachine Aufbau II}
  \vspace{-.7cm}
  \begin{lstlisting}
state_v := state_cs;
case state_cs is
  when "00" =>
    -- phase 1

  when "01" =>
    -- phase 2

  when "10" =>
    -- phase 3

  when others =>
    -- default case
    
end case;
state_ns <= state_v;
  \end{lstlisting}
\end{frame}


\begin{frame} {State 1}
  \begin{itemize}
    \item Read?
    \begin{itemize}
      \item oe $\leftarrow$ 1
      \item dato $\leftarrow$ fx
    \end{itemize}
    \item Write?
    \begin{itemize}
      \item oe $\leftarrow$ 0
      \item req $\leftarrow$ 1
      \item rdy $\leftarrow$ 0
    \end{itemize}
    \item state $\leftarrow$ 1
  \end{itemize}
\end{frame}


\begin{frame} {State 2}
  \begin{itemize}
    \item ack $\leftarrow$ 1
    \item req = 0?
    \begin{itemize}
      \item oe $\leftarrow$ 0
      \item state $\leftarrow$ 2
    \end{itemize}
  \end{itemize}
\end{frame}


\begin{frame} {State 3}
  \begin{itemize}
    \item ack $\leftarrow$ 0
    \item oe $\leftarrow$ 0
    \item state $\leftarrow$ 0
  \end{itemize}
\end{frame}


\begin{frame} {Aufbau Der Schaltung}
  \begin{center}
    \begin{tikzpicture} [    
      auto,
      block/.style={
      rectangle,
      draw=blue,
      thick,
      fill=blue!20,
      text width=5em,
      text centered,
      align=center,
      rounded corners,
      minimum height=2em
      }]

      \node[block, text centered,text width = 0.25\textwidth, minimum height=5cm, draw] (stm) {STM32};
    
      \node[block, text width=3cm] (controller) [right=of stm, xshift=5cm, yshift=2cm] {Controller};
      \node[block, text centered,text width = 0.25\textwidth, minimum height = 3 cm, draw] (computer) [below=of controller]{sqrtComputer};
      \node[draw,thick,rounded corners, draw=blue,fit=(controller) (computer)] {};
      
      \path[every node/.style={sloped,anchor=south,auto=false}]
        (stm) edge node [above] {REQ, ACK} node [below] {RD/nWR, DATA} (controller)

        (controller) edge [transform canvas={xshift=-1cm}]  node {req} (computer)            
        (controller) edge [transform canvas={xshift=-.3cm}] node {fx} (computer)
        (controller) edge [transform canvas={xshift=.3cm}]  node {x} (computer)
        (controller) edge [transform canvas={xshift=1cm}]  node {rdy} (computer);

    \end{tikzpicture}
  \end{center}
\end{frame}

\end{document}