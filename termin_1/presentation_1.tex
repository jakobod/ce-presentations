%%% Globale Einstellungen und Laden von Paketen (~Bibliotheken)
\documentclass[aspectratio=169,presentation]{beamer}
%\documentclass[aspectratio=169,handout]{beamer}

\usetheme{Boadilla} % Bestimmt das gesamte Erscheinungsbild, die folgenden fand ich grundsätzlich ganz passend:
% Hannover, Singapore, Malmoe, Boadilla, CambridgeUS

\useinnertheme{default}

% setting locales
\usepackage[utf8]{inputenc}
\usepackage[T1]{fontenc}
\usepackage[ngerman]{babel}
\usepackage{lmodern}
\usepackage[locale=DE,mode=math,list-final-separator={ oder },range-phrase={ bis },scientific-notation=false,group-digits=integer]{siunitx}

% package includes
\usepackage{tikz}
\usetikzlibrary{positioning,automata}
\usepackage{xcolor}
\usepackage{listings}

%listing setup
\definecolor{pblue}{rgb}{0.13,0.13,1}
\definecolor{pgreen}{rgb}{0,0.5,0}
\definecolor{pred}{rgb}{0.9,0,0}
\definecolor{pgrey}{rgb}{0.46,0.45,0.48}
\definecolor{javared}{rgb}{0.6,0,0} % for strings
\definecolor{javagreen}{rgb}{0.25,0.5,0.35} % comments
\definecolor{javapurple}{rgb}{0.5,0,0.35} % keywords
\definecolor{javadocblue}{rgb}{0.25,0.35,0.75} % javadoc

\lstset{language=vhdl,
	basicstyle=\ttfamily,
	keywordstyle=\color{javapurple}\bfseries,
	stringstyle=\color{javared},
	commentstyle=\color{javagreen},
	morecomment=[s][\color{javadocblue}]{/**}{*/},
	tabsize=2,
	showspaces=false,
	showstringspaces=false
}


%%% (Wahrscheinlich ziemlich dreckige) Umsetzung von Spezialframes, die nur groß den Titel beinhalten
\newcommand{\sectionframe}[1]{
	\begin{frame}
		\vfill
		\Huge
		\centering
		\usebeamercolor[fg]{title}
		#1
		\vfill
		\par
	\end{frame}
}



%%% Zentrales Festelegen von Terminnummer und Datum
\newcommand{\terminNummer}{2}
\date{\today}
%%%



\begin{document}
	\title[DT Tutorium]{Tutorium zu\\Computer-Engineering\\im SS19}
	\subtitle{Termin \terminNummer}
	\author[Otto]{Jakob Otto}
	\institute{HAW Hamburg}
	\subject{CE Tutorium}
	\pgfdeclareimage[height=0.5cm]{university-logo}{logo-haw-2017}
	\logo{\href{http://haw-hamburg.de}{\pgfuseimage{university-logo}}}
	
	\begin{frame}
		\titlepage
	\end{frame}

%---------------------------------------------------------------------------------------------------------------------
%	Ablauf
%---------------------------------------------------------------------------------------------------------------------
\section{Was steht an?}
\begin{frame}{Ablauf}
	\begin{columns}
		\column{0.6\textwidth}
		\begin{itemize}
			\item Praktikum
			\begin{itemize}
				\item wdh. UCF-Files
				\item tick-generator
				\item shiftregister
				\item lookup table
			\end{itemize}
		\end{itemize}
		\column{0.4\textwidth}
		\includegraphics[width=0.6\textwidth]{kratzen}
	\end{columns}
\end{frame}

%---------------------------------------------------------------------------------------------------------------------
%	Ideen für Aufgabe 1
%---------------------------------------------------------------------------------------------------------------------
%	- Ideen für Tick-generator zeigen
%	- Lookup-tables ansprechen
%	- 
%
%
%

\sectionframe{\href{http://users.informatik.haw-hamburg.de/~schafers/LOCAL/S19S_CE/Aufgabenzettel_Nr1_v15.pdf}{Aufgabenzettel}}

\section{UCF-Datei}
\sectionframe{UCF-Datei}
\begin{frame} [fragile] {kleine Wiederholung}
	\begin{itemize}
		\item UCF-Datei mappt ein- und Ausgänge auf pins
		\item Format:
		\begin{lstlisting}
NET <port-name> LOC = <pin> | IOSTANDARD=LVCMOS33;
		\end{lstlisting}
		\item Als Beispiel:
		\begin{lstlisting}
NET "clk"		LOC = "V10" | IOSTANDARD=LVCMOS33;
		\end{lstlisting}
	\end{itemize}
\end{frame}

\begin{frame} [fragile] {Woher die ganzen pins?!}
	\begin{itemize}
		\item Normalerweise aus irgendwelchen Datenblättern..
		\item Hier: Aus der 
		\href{https://users.informatik.haw-hamburg.de/~behn/pub/CEP/FPGA_CE Board.pdf}{\color{red}{CE\underline{ }Board-Doku}}
		\item Weitere Doku zum Nexys2-board gibts \href{http://users.informatik.haw-hamburg.de/~schafers/LOCAL/S19S_CE/DOCU/OLD Digilent Nexys2 Board Reference Manual.pdf}{\color{red}{HIER}}
	\end{itemize}
\end{frame}


\section{Ideen zur Praktikumsaufgabe}
\sectionframe{Ideen zur Praktikumsaufgabe}
\begin{frame} {Tick-Generator}
	\begin{itemize}
		\item erzeugt ticks mit einer gewünschten Frequenz auf Basis eines gegebenen Taktes
		\item kann Frequenz nur absenken!
		\item im Grunde zählt der Tickgenerator nur Takte\\
		\item Ticks werden wie ein druck eines buttons gewertet.
		\begin{itemize}
			 \item Enable-Signal
		\end{itemize}
	\end{itemize}
\end{frame}


\begin{frame} [fragile] {Tickgenerator-idee I}
	\begin{lstlisting}
entity TickGen is
    port(
        tick : out std_logic;
        clk : in std_logic
    );
end entity TickGen;
	\end{lstlisting}
\end{frame}


\begin{frame} [fragile] {Tickgenerator-idee II}
	\begin{lstlisting}
tickGen: process (clk) is
	constant maxValue : integer := XXXXX;
	variable count : integer range 0 to maxValue := 0; -- nicht gut!
	variable tick_v : std_logic;
begin
	if (rising_edge(clk)) then -- geht wohl noch
		count := count + 1;
		if (count = maxValue) then
			count := 0;
			tick_v := '1';
		else
			tick_v := '0';
		end if;
	end if;
	tick <= tick_v;
end process tickGen;
	\end{lstlisting}
\end{frame}


\begin{frame} {Auswahl der Anoden}
	\begin{itemize}
		\item Die Anoden müssen zyklisch mit durschgewechselt werden
		\item Dazu nutzt ein zyklisches shiftregister!
		\item shiftregister soll '1' mit frequenz des generierten ticks shiften.
		\item Dadurch werden nacheinander die einzelnen Anoden durchgeschaltet.
	\end{itemize}
\end{frame}


\begin{frame} [fragile]
	\begin{lstlisting}
sequlo:
process (clk) is
begin
	-- kennt ihr..
end process sequlo;

shift:
process(shiftRegister_cs) is 
begin
	shiftRegister_ns <= shiftRegister_cs(shiftRegister_cs'left-1 downto 0) & shiftRegister_cs(shiftRegister_cs'left); -- shift all bits to the left and add the old MSB to the right again.
end process shift;
	\end{lstlisting}
	shiftregister sollte mit \grqq{}1110\grqq{} initialisiert werden!\\
	$\rightarrow$ Anode ist low-aktiv
\end{frame}


\begin{frame} {Lookup table}
	\begin{itemize}
		\item Die nibble werden mit 4 bit codiert
		\item zum Anzeigen müssen die nibble auf die Kathodenausgänge umgesetzt werden.
		\item Die Ausgangswerte werden dazu in einem lookup-table hinterlegt werden
		\item $\rightarrow$ case-when!
	\end{itemize}
\end{frame}

\begin{frame} [fragile] {Lookup table}
	\begin{lstlisting}
case nibVal_v is
	when "0000" => 
		segments_v := "10000001";
	when "0001" => 
		segments_v := "11001111";
	when "0010" => 
		segments_v := "10010010";    
	-- usw
end case;
	\end{lstlisting}
\end{frame}



\end{document}