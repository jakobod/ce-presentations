%%% Globale Einstellungen und Laden von Paketen (~Bibliotheken)
\documentclass[aspectratio=169,presentation]{beamer}
%\documentclass[aspectratio=169,handout]{beamer}

\usetheme{Boadilla} % Bestimmt das gesamte Erscheinungsbild, die folgenden fand ich grundsätzlich ganz passend:
% Hannover, Singapore, Malmoe, Boadilla, CambridgeUS

\useinnertheme{default}

% setting locales
\usepackage[utf8]{inputenc}
\usepackage[T1]{fontenc}
\usepackage[ngerman]{babel}
\usepackage{lmodern}
\usepackage[locale=DE,mode=math,list-final-separator={ oder },range-phrase={ bis },scientific-notation=false,group-digits=integer]{siunitx}

% package includes
\usepackage{tikz}
\usetikzlibrary{positioning,automata}
\usepackage{xcolor}
\usepackage{listings}

%listing setup
\definecolor{pblue}{rgb}{0.13,0.13,1}
\definecolor{pgreen}{rgb}{0,0.5,0}
\definecolor{pred}{rgb}{0.9,0,0}
\definecolor{pgrey}{rgb}{0.46,0.45,0.48}
\definecolor{javared}{rgb}{0.6,0,0} % for strings
\definecolor{javagreen}{rgb}{0.25,0.5,0.35} % comments
\definecolor{javapurple}{rgb}{0.5,0,0.35} % keywords
\definecolor{javadocblue}{rgb}{0.25,0.35,0.75} % javadoc

\lstset{language=vhdl,
	basicstyle=\ttfamily,
	keywordstyle=\color{javapurple}\bfseries,
	stringstyle=\color{javared},
	commentstyle=\color{javagreen},
	morecomment=[s][\color{javadocblue}]{/**}{*/},
	tabsize=2,
	showspaces=false,
	showstringspaces=false
}


%%% (Wahrscheinlich ziemlich dreckige) Umsetzung von Spezialframes, die nur groß den Titel beinhalten
\newcommand{\sectionframe}[1]{
	\begin{frame}
		\vfill
		\Huge
		\centering
		\usebeamercolor[fg]{title}
		#1
		\vfill
		\par
	\end{frame}
}



%%% Zentrales Festelegen von Terminnummer und Datum
\newcommand{\terminNummer}{2}
\date{\today}
%%%



\begin{document}
\title[CE Tutorium]{Tutorium zu\\Computer-Engineering\\im SS19}
\subtitle{Termin \terminNummer}
\author[Otto]{Jakob Otto}
\institute{HAW Hamburg}
\subject{CE Tutorium}
\pgfdeclareimage[height=0.5cm]{university-logo}{logo-haw-2017}
\logo{\href{http://haw-hamburg.de}{\pgfuseimage{university-logo}}}

\begin{frame}
	\titlepage
\end{frame}

%---------------------------------------------------------------------------------------------------------------------
%	Ablauf
%---------------------------------------------------------------------------------------------------------------------
\section{Was steht an?}
\begin{frame}{Ablauf}
	\begin{columns}
		\column{0.6\textwidth}
		\begin{itemize}
			\item Praktikum
			\begin{itemize}
				\item Ideen zum Aufbau
				\item Trial-subtraction-Verfahren
				\item Testen
			\end{itemize}
		\end{itemize}
		\column{0.4\textwidth}
		\includegraphics[width=0.6\textwidth]{kratzen}
	\end{columns}
\end{frame}

%---------------------------------------------------------------------------------------------------------------------
%	Ideen für Aufgabe 1
%---------------------------------------------------------------------------------------------------------------------
%	- Trial-subtraction verfahren


\sectionframe{\href{http://users.informatik.haw-hamburg.de/~schafers/LOCAL/S19S_CE/Aufgabenzettel_Nr2_v08.pdf}{Aufgabenzettel}}


\section{Ideen zur Aufgabe}
\begin{frame} {Addierer/Subtrahierer?!}
	\begin{itemize}
		\item Ihr dürft \underline{einen} Addierer und \underline{einen} Subtrahierer nutzen.
		\begin{itemize}
			\item der Addierer ist für das Inkrementieren des Zustands gedacht
			\item der Subtrahierer für den Algorithmus
		\end{itemize}
		\item Denkt an Aufgabe 3 aus DT und steuert den Addierer/Subtrahierer.		      
	\end{itemize}
\end{frame}


\begin{frame} [fragile] {Addierer/Subtrahierer?!}
\textbf{NICHT!!}
	\begin{lstlisting}
something: process (...) is 
	-- paar variablen
begin
	if (select_v = '0') then
		res_v := a + b;
	else
		res_v := b + c;
	end if;
	
	-- ggf. interpretation

	res_s <= res_v;
end process;
	\end{lstlisting}
\end{frame}


\begin{frame} [fragile] {Addierer/Subtrahierer?!}
\textbf{BESSER.}
	\begin{lstlisting}
something: process (...) is 
	-- paar variablen
begin
	if (select_v = '0') then
		op_a := a;
		op_b := b;
	else
		op_a := b;
		op_b := c;
	end if;
	res_v := op_a + op_b

	-- ggf. interpretation. 
	res_s <= res_v;
end process;
	\end{lstlisting}
\end{frame}


\begin{frame} {kleiner Tipp zum Code}
	\begin{itemize}
		\item Versucht \textbf{NICHT} den Code modular zu gestalten
		\item d.H. schreibt einen (zwei mit sequlo) Prozess, der die State machine beinhaltet
		\item Modularisieren ist eine gute Sache, aber macht das Ziel von 40 Zuständen schwer erreichbar..
	\end{itemize}
\end{frame}


\begin{frame} {Eingaben}
	\begin{itemize}
		\item Bei Eingabe von:
		\begin{itemize}
			\item -1 $\rightarrow$ "1000000000000000"
			\item 0 $\rightarrow$ "0000000000000000"
		\end{itemize}
		braucht ihr keine Berechnung zu starten. Eingabe einfach als Ergebnis durchreichen
	\end{itemize}	
\end{frame}


\begin{frame} {Trial-subtraction}
	Definitionen der Variablen:
	\begin{itemize}
		\item $u$ $\rightarrow$ \textbf{positive} Zahl (die Eingabe)
		\item $s$ $\rightarrow$ Approximation der Wurzel
		\item $s'$ $\rightarrow$ die \grqq{}neue\grqq{} Approximation
		\item $r$ $\rightarrow$ der mögliche Rest
		\item $r'$ $\rightarrow$ ein möglicher \grqq{}anderer\grqq{} Rest
		\item $d$ $\rightarrow$ Differenz -> $2^n$
	\end{itemize}
\end{frame}

\begin{frame} {Trial-subtraction}
	Formeln: \\
	$r' = r - 2^n (2s + 2^n)$ \\
	$s' = s + 2^n$ $\leftarrow$ Hierzu gleich mehr!
\end{frame}

\begin{frame}
	Algorithmus:
	\begin{enumerate}
		\item $r'$ berechnen
		\item wenn:
		\begin{itemize}
			\item r' >= 0 $\rightarrow$ s' berechnen und s = s'
			\item r' < 0 $\rightarrow$ s = s bleibt also.
		\end{itemize}
		\item sobald r' = 0 $\rightarrow$ ENDE
	\end{enumerate}
\end{frame}


\begin{frame} [fragile] {und nun in code?!}
	Denkt an: \\
	$r' = r - 2^n (2s + 2^n)$
	\vspace{0.3cm}
	\begin{lstlisting}
delta_v := to_stdlogicvector("1000000000000000" srl n_v);
operandB_v := to_stdlogicvector(to_bitvector(
				(s_v(msbPos-1 downto 0) & '0') or (delta_v)) srl n_v);
operandA_v := r_v;
result_v := opA_v - opB_v;

-- weiter interpretieren
	\end{lstlisting}
\end{frame}


\begin{frame}
	wichtig:
	\begin{itemize}
		\item Bevor ihr eine Berechnung startet:
		\item Eingabe auf VZ prüfen und ggf. positiv machen. $\rightarrow (0-Wert)$ \textbf{VZ merken!}
		\item Nach der Berechnung Ergebnis wieder Negativ machen $\rightarrow (0-Ergebnis)$
	\end{itemize}
\end{frame}


\section{Testen}
\begin{frame} [fragile] {Testframe}
	\begin{lstlisting}
for i in -32768 to 32767 loop
	x_s <= std_logic_vector(to_signed(i, x_s'length));
	req_s <= '1'; 

	-- warten, dass Berechnung gestartet wurde
	wait for fullClockCycle;
	req_s <= '0';

	-- warten auf Ende der Berechnung
	wait for 18 * fullClockCycle;
end loop;
	\end{lstlisting}
\end{frame}

\end{document}