% include all necessary latex libraries
\documentclass[aspectratio=169,presentation]{beamer}
\usetheme{Boadilla}
\useinnertheme{default}

% -- locale-setup --------------------------------------------------------------

\usepackage[utf8]{inputenc}
\usepackage[T1]{fontenc}
\usepackage[ngerman]{babel}
\usepackage{lmodern}
\usepackage[locale=DE, mode=math, list-final-separator={ oder },
  range-phrase={ bis }, scientific-notation=false,
  group-digits=integer] {siunitx}

% -- package-includes ----------------------------------------------------------

\usepackage{tikz}
\usepackage{xcolor}
\usetikzlibrary{positioning,automata,matrix,arrows.meta,fit}
\usepackage{graphicx} % including images
\usepackage[font=small,labelfont=bf]{caption} % captions of tables and figures

% -- listing setup -------------------------------------------------------------

\usepackage{listings}
\definecolor{pblue}{rgb}{0.13,0.13,1}
\definecolor{pgreen}{rgb}{0,0.5,0}
\definecolor{pred}{rgb}{0.9,0,0}
\definecolor{pgrey}{rgb}{0.46,0.45,0.48}
\definecolor{javared}{rgb}{0.6,0,0} % for strings
\definecolor{javagreen}{rgb}{0.25,0.5,0.35} % comments
\definecolor{javapurple}{rgb}{0.5,0,0.35} % keywords
\definecolor{javadocblue}{rgb}{0.25,0.35,0.75} % javadoc

\lstset{language=vhdl,
  basicstyle=\ttfamily,
  keywordstyle=\color{javapurple}\bfseries,
  stringstyle=\color{javared},
  commentstyle=\color{javagreen},
  morecomment=[s][\color{javadocblue}]{/**}{*/},
  tabsize=2,
  showspaces=false,
  showstringspaces=false
}

% -- custom commands -----------------------------------------------------------

% just prints given text in large text and on empty frame.
\newcommand{\sectionframe} [1] {
  \begin{frame}
    \vfill
    \Huge
    \centering
    \usebeamercolor[fg]{title}
    #1
    \vfill
    \par
  \end{frame}
}

% wrapper-command to make pretty title takes
% 1=short-title 2=long-title 3=terminnumber 4=date
\newcommand{\maketitlepage} [5] {
  \title[#1]{#2}
  \subtitle{Termin #3}
  \date{#4}
  \author[Jakob Otto]{Jakob Otto}
  \institute{HAW Hamburg}
  \subject{#1}
  \pgfdeclareimage[height=0.5cm, width=1.3cm]{university-logo}{#5}
  \logo{\href{https://www.haw-hamburg.de}{\pgfuseimage{university-logo}}}
  \begin{frame} {}
    \titlepage
  \end{frame}
}

% draws a link symbol
\newcommand{\externalLink} {
  \tikz[x=1.2ex, y=1.2ex, baseline=-0.05ex]{
    \begin{scope}[x=1ex, y=1ex]
      \clip (-0.1,-0.1)
      --++ (-0, 1.2)
      --++ (0.6, 0)
      --++ (0, -0.6)
      --++ (0.6, 0)
      --++ (0, -1);
      \path[draw,
        line width = 0.5,
        rounded corners=0.5]
      (0,0) rectangle (1,1);
    \end{scope}
    \path[draw, line width = 0.5] (0.5, 0.5)
    -- (1, 1);
    \path[draw, line width = 0.5] (0.6, 1)
    -- (1, 1) -- (1, 0.6);
  }
}

% Adds a circled number -> Looks better than the default
\newcommand{\circled} [1] {
  \tikz[baseline=(char.base)]{
    \node[shape=circle,draw,inner sep=2pt, text centered, text width = .2cm] (char) {#1};
  }
}

\newcommand{\link} [2] {
  \href{#1}{#2\externalLink}
}

% -- extra includes go here ----------------------------------------------------
\usepackage{tikz}
\usepackage[customcolors]{hf-tikz}
\usetikzlibrary{shapes,arrows, positioning, fit, backgrounds}

\newcommand{\connectlr} [3] {
  \path (#1.east) -- (#1.north east) coordinate[pos=0.5] (tmp11);
  \path (#2.west) -- (#2.north west) coordinate[pos=0.5] (tmp21);
  \draw[-latex] (tmp21) -- (tmp11);
  \path (#1.east) -- (#1.south east) coordinate[pos=0.5] (tmp12);
  \path (#2.west) -- (#2.south west) coordinate[pos=0.5] (tmp22);
  \draw[-latex] (tmp12) -- (tmp22);
  \path (#1) -- (#2) node [midway] {\circled{#3}};
}

\newcommand{\connecttb} [3] {  
  \path (#1.south) -- (#1.south west) coordinate[pos=0.5] (tmp11);
  \path (#2.north) -- (#2.north west) coordinate[pos=0.5] (tmp21);
  \draw[-latex] (tmp21) -- (tmp11);
  \path (#1.south) -- (#1.south east) coordinate[pos=0.5] (tmp12);
  \path (#2.north) -- (#2.north east) coordinate[pos=0.5] (tmp22);
  \draw[-latex] (tmp12) -- (tmp22);
  \path (#1) -- (#2) node [midway] {\circled{#3}};
}

% put color to \boxed math command
\newcommand*{\boxcolor}{orange}
\makeatletter
\renewcommand{\boxed}[2]{\textcolor{#1}{%
\tikz[baseline={([yshift=-1ex]current bounding box.center)}] \node [rectangle, minimum width=1ex,rounded corners,draw] {\normalcolor\m@th$\displaystyle#2$};}}
 \makeatother

%-------------------------------------------------------------------------------
%	Variablen
%-------------------------------------------------------------------------------

\newcommand{\terminnumber}{4}
\newcommand{\hawlogo}{../presentation-template/figs/logo-haw-2017}
\newcommand{\kratzen}{../presentation-template/figs/kratzen}
\newcommand{\aufgabenzettellink}{https://users.informatik.haw-hamburg.de/~schafers/LOCAL/S19W_CE/Aufgabenzettel_Nr2_v10.pdf}

%-------------------------------------------------------------------------------
%	Dokument
%-------------------------------------------------------------------------------

\begin{document}
\maketitlepage {CE Tutorium} {Tutorium zu\\Computer-Engineering\\im WS19}
{\terminnumber} {\today} {\hawlogo}

%-------------------------------------------------------------------------------
%	Ablauf
%-------------------------------------------------------------------------------

\section{Was steht an?}
\begin{frame}{Ablauf}
  \begin{columns}
    \column{0.6\textwidth}
    \begin{itemize}
      \item Praktikum
            \begin{itemize}
              \item Was ist zu tun?
              \item Was braucht ihr?
              \item Beispielcode
              \item Tipps
            \end{itemize}
    \end{itemize}
    \column{0.4\textwidth}
    \includegraphics[width=0.6\textwidth]{kratzen}
  \end{columns}
\end{frame}

%-------------------------------------------------------------------------------
%	Ausblick
%-------------------------------------------------------------------------------

\section{Ausblick}
\begin{frame} {Ausblick (I)}
  \begin{block} {Ablauf}
    \begin{enumerate}
      \item Trial-Subtraction Algorithmus
            \begin{itemize}
              \item effizientes Wurzelziehen aus Samples
            \end{itemize}
      \item DAC spielereien
            \begin{itemize}
              \item Ausgabe von Sound lernen
            \end{itemize}
      \item Flash-speicher lesen/schreiben
            \begin{itemize}
              \item Samples lesen lernen
            \end{itemize}
      \item Alles zusammensetzen
            \begin{itemize}
              \item Kommunikation zwischen FPGA/STM-32
              \item Ausgabe übr PWM
              \item Musik abspielen
            \end{itemize}
    \end{enumerate}
  \end{block}
\end{frame}

\begin{frame} {Ausblick (II)}
  % Define a few styles and constants
  \tikzstyle{block} = [text width=6em, text centered, minimum height=6em, 
  rounded corners, draw]
  \tikzstyle{stm32} = [fill=red!20, block]
  \tikzstyle{fpga}=[fill=blue!20, block]
  \tikzstyle{flash}=[fill=green!20, block]
  \tikzstyle{output}=[fill=yellow!20, block]
  \begin{center}
    \begin{tikzpicture} [node distance=5cm]
      \node (stm32) [stm32] {STM-32};
      \node (fpga) [fpga, left of = stm32] {FPGA};
      \node (flash) [flash, below of = stm32] {flash-memory};
      \node (output) [output, right of = stm32] {PWM/\\DAC};
      \node (legend) [below of=output,fill=white, draw, text width=5cm] {
        \circled{1} Samples beschaffen\\
        \circled{2} Wurzel ziehen\\
        \circled{3} ausgeben
      };
      % -- connect fpga and stm32 --------------------------------------------
      \connectlr {fpga} {stm32} {2}
      \connectlr {stm32} {output} {3}
      \connecttb {stm32} {flash} {1}
    \end{tikzpicture}
  \end{center}
\end{frame}

\begin{frame} {Ausblick (III)}
  \tikzstyle{block} = [text width=6em, text centered, minimum height=6em, 
  rounded corners, draw]
  \tikzstyle{stm32} = [fill=red!20, block]
  \tikzstyle{fpga}=[fill=blue!20, block]
  \tikzstyle{flash}=[fill=green!20, block]
  \tikzstyle{output}=[fill=yellow!20, block]
  \tikzstyle{surround} = [fill=white,thick,draw=red,rounded corners=2mm]
  \begin{center}
    \begin{tikzpicture} [node distance=5cm]
      \node (stm32) [stm32] {STM-32};
      \node (fpga) [fpga, left of = stm32] {FPGA};
      \node (flash) [flash, below of = stm32] {flash-memory};
      \node (output) [output, right of = stm32] {PWM/\\DAC};
      \node (legend) [below of=output,fill=white, draw, text width=5cm] {
        \circled{1} Samples beschaffen\\
        \circled{2} Wurzel ziehen\\
        \circled{3} ausgeben
      };
      \begin{pgfonlayer}{background}
        \node[surround] (background) [fit = (stm32) (output)] {};
      \end{pgfonlayer}
      % -- connect fpga and stm32 --------------------------------------------
      \connectlr {fpga} {stm32} {2}
      \connectlr {stm32} {output} {3}
      \connecttb {stm32} {flash} {1}
    \end{tikzpicture}
  \end{center}
\end{frame}

%-------------------------------------------------------------------------------
%	Praktikum
%-------------------------------------------------------------------------------

\section*{Praktikum}
\sectionframe{\link{http://users.informatik.haw-hamburg.de/~schafers/LOCAL/S19S_CE/Aufgabenzettel_Nr3_v06_Entwurf.pdf}{Aufgabenzettel}}
\begin{frame} {Praktikum (I)}
  \begin{block} {Was ist das Ziel?}
    \begin{itemize}
      \item DAC verstehen!
      \item Sinus/Sägezahnsignale ausgeben
      \item verschiedene Frequenzen darstellen
      \item verschiedene Amplituden darstellen
    \end{itemize}
  \end{block}
\end{frame}

% Schäfers Beispielcode
\sectionframe{\link{https://users.informatik.haw-hamburg.de/~schafers/LOCAL/S19S_CE/CODE/DemoForLabA3_Curve/main.c}{Schäfers Beispielcode}}

\begin{frame} {Praktikum (II)}
  \begin{block} {Was Passiert da?}
    Zwei verschiedene Handlungsstränge!
    \begin{enumerate}
      \item Hauptroutine
            \begin{itemize}
              \item Pollt buttons
              \item Füllt den buffer mit samples
              \item Die main halt
            \end{itemize}
      \item ISR
            \begin{itemize}
              \item Wird durch Interrupts ausgelöst
              \item Schreibt nächstes sample in DAC-Register
            \end{itemize}
    \end{enumerate}
  \end{block}
\end{frame}

\sectionframe {init}

\begin{frame} {Init }
  
\end{frame}

\begin{frame} {Praktikum (II)}
  \begin{block} {Samples?}
    Für die Lookup-tables braucht ihr Samples.
    \begin{itemize}
      \item volle Periode des Signals berechnen
      \item Samples in einem Array hard-coden
      \item Am besten ohne Offset speichern $\rightarrow$ Signal sollte um 0-pkt laufen.
      \item erst beim nutzen geeignet umformen.
    \end{itemize}
  \end{block}
\end{frame}

\begin{frame} [fragile] {Wie Umformen?}
  \begin{itemize}
    \item Samples z.B. in +1V/-1V Format speichern
    \item Beim nutzen dann teilen
  \end{itemize}
  \begin{lstlisting}
// darstellung +1V/-1V
fifo[index] = samples[sampleIndex] + offset;
// Darstellung +0.5V/-0.5V
fifo[index] = (samples[sampleIndex] >> 1) + offset;
  \end{lstlisting}
\end{frame}

\sectionframe{Berechnungsbeispiel}

\begin{frame} {Berechnung der Schrittweite}
  Zum Darstellen verschiedener Frequenzen benötigt ihr verschiedene Schrittweiten.
  \begin{itemize}
    \item kleine Schrittweite $\rightarrow$ kleine Frequenz
    \item große Schrittweite $\rightarrow$ große Frequenz
  \end{itemize}
  \vspace{0.5cm}
  Berechnung:\\
  $delta_{freq} = ((((ANZ\_SAMPLES) * FREQ) << frac) / TIMER\_FREQ)$\\
  frac = fractional Anteil des Q-Formats
\end{frame}

\begin{frame} {Schrittweite $\rightarrow$ Q-Format?!}
  Für höchste genauigkeit Q-Format nutzen!\\
  \vspace{0.5cm}
  \pause
  Bei 360 samples brauchen wir 9 Integer-bits $\rightarrow$ $2^9 = 512$ \\
  \vspace{0.5cm}
  Qu9.23 ist also sinnvolles Format
\end{frame}


\end{document}