%%% Globale Einstellungen und Laden von Paketen (~Bibliotheken)
\documentclass[aspectratio=169,presentation]{beamer}
%\documentclass[aspectratio=169,handout]{beamer}

\usetheme{Boadilla} % Bestimmt das gesamte Erscheinungsbild, die folgenden fand ich grundsätzlich ganz passend:
% Hannover, Singapore, Malmoe, Boadilla, CambridgeUS

\useinnertheme{default}

% setting locales
\usepackage[utf8]{inputenc}
\usepackage[T1]{fontenc}
\usepackage[ngerman]{babel}
\usepackage{lmodern}
\usepackage[locale=DE,mode=math,list-final-separator={ oder },range-phrase={ bis },scientific-notation=false,group-digits=integer]{siunitx}

% package includes
\usepackage{tikz}
\usetikzlibrary{positioning,automata}
\usepackage{xcolor}
\usepackage{listings}

%listing setup
\definecolor{pblue}{rgb}{0.13,0.13,1}
\definecolor{pgreen}{rgb}{0,0.5,0}
\definecolor{pred}{rgb}{0.9,0,0}
\definecolor{pgrey}{rgb}{0.46,0.45,0.48}
\definecolor{javared}{rgb}{0.6,0,0} % for strings
\definecolor{javagreen}{rgb}{0.25,0.5,0.35} % comments
\definecolor{javapurple}{rgb}{0.5,0,0.35} % keywords
\definecolor{javadocblue}{rgb}{0.25,0.35,0.75} % javadoc

\lstset{language=c,
	basicstyle=\ttfamily,
	keywordstyle=\color{javapurple}\bfseries,
	stringstyle=\color{javared},
	commentstyle=\color{javagreen},
	morecomment=[s][\color{javadocblue}]{/**}{*/},
	tabsize=2,
	showspaces=false,
	showstringspaces=false
}


%%% (Wahrscheinlich ziemlich dreckige) Umsetzung von Spezialframes, die nur groß den Titel beinhalten
\newcommand{\sectionframe}[1]{
	\begin{frame}
		\vfill
		\Huge
		\centering
		\usebeamercolor[fg]{title}
		#1
		\vfill
		\par
	\end{frame}
}



%%% Zentrales Festelegen von Terminnummer und Datum
\newcommand{\terminNummer}{4}
\date{\today}
%%%



\begin{document}
\title[CE Tutorium]{Tutorium zu\\Computer-Engineering\\im SS19}
\subtitle{Termin \terminNummer}
\author[Otto]{Jakob Otto}
\institute{HAW Hamburg}
\subject{CE Tutorium}
\pgfdeclareimage[height=0.5cm]{university-logo}{logo-haw-2017}
\logo{\href{http://haw-hamburg.de}{\pgfuseimage{university-logo}}}

\titlepage

%---------------------------------------------------------------------------------------------------------------------
%	Ablauf
%---------------------------------------------------------------------------------------------------------------------
\section{Was steht an?}
\begin{frame}{Ablauf}
	\begin{columns}
		\column{0.6\textwidth}
		\begin{itemize}
			\item Praktikum
      \begin{itemize}
        \item Was ist zu tun?
        \item Was braucht ihr?
        \item Beispielcode
				\item Tipps
			\end{itemize}
		\end{itemize}
		\column{0.4\textwidth}
		\includegraphics[width=0.6\textwidth]{kratzen}
	\end{columns}
\end{frame}

%---------------------------------------------------------------------------------------------------------------------
%	Ideen für Aufgabe 1
%---------------------------------------------------------------------------------------------------------------------
%	- Trial-subtraction verfahren


\sectionframe{\href{http://users.informatik.haw-hamburg.de/~schafers/LOCAL/S19S_CE/Aufgabenzettel_Nr3_v06_Entwurf.pdf}{Aufgabenzettel}}

\begin{frame} {Was ist zu tun?}
  \begin{itemize}
    \item DAC verstehen!
    \item Sinus/Sägezahnsignale ausgeben
    \item verschiedene Frequenzen darstellen
    \item verschiedene Amplituden darstellen
  \end{itemize}
\end{frame}

% Beispielcode
\sectionframe{\href{https://users.informatik.haw-hamburg.de/~schafers/LOCAL/S19S_CE/CODE/DemoForLabA3_Curve/main.c}{Beispielcode}}

\begin{frame} {Wie kommt ihr an Samples?!}
  Für die Lookup-tables braucht ihr Samples.
  \begin{itemize}
    \item volle Periode des Signals berechnen
    \item Samples in einem Array hard-coden
    \item Am besten ohne Offset speichern $\rightarrow$ Signal sollte um 0-pkt laufen.
    \item erst beim nutzen geeignet umformen. 
  \end{itemize}
\end{frame}

\begin{frame} [fragile] {Wie Umformen?}
  \begin{itemize}
    \item Samples z.B. in +1V/-1V Format speichern
    \item Beim nutzen dann teilen
  \end{itemize}
  \begin{lstlisting}
// darstellung +1V/-1V
fifo[index] = samples[sampleIndex] + offset;
// Darstellung +0.5V/-0.5V
fifo[index] = (samples[sampleIndex] >> 1) + offset;
  \end{lstlisting}
\end{frame}

\sectionframe{Berechnungsbeispiel}

\begin{frame} {Berechnung der Schrittweite}
  Zum Darstellen verschiedener Frequenzen benötigt ihr verschiedene Schrittweiten.
  \begin{itemize}
    \item kleine Schrittweite $\rightarrow$ kleine Frequenz
    \item große Schrittweite $\rightarrow$ große Frequenz
  \end{itemize}
  \vspace{0.5cm}
  Berechnung:\\
  $delta_{freq} = ((((ANZ\_SAMPLES) * FREQ) << frac) / TIMER\_FREQ)$\\
  frac = fractional Anteil des Q-Formats
\end{frame}

\begin{frame} {Schrittweite $\rightarrow$ Q-Format?!}
  Für höchste genauigkeit Q-Format nutzen!\\
  \vspace{0.5cm}
  \pause
  Bei 360 samples brauchen wir 9 Integer-bits $\rightarrow$ $2^9 = 512$ \\
  \vspace{0.5cm}
  Qu9.23 ist also sinnvolles Format
\end{frame}


\end{document}